\usepackage{setspace}
\usepackage{wasysym}
% \usepackage{footnote} % don't use this this breaks all
\usepackage{fontenc}
\usepackage{fontawesome}
\usepackage{booktabs,siunitx}
\usepackage{longtable}
\usepackage{array}
\usepackage{multirow}
\usepackage{wrapfig}
\usepackage{float}
\usepackage{colortbl}
\usepackage{pdflscape}
\usepackage{tabu}
\usepackage{threeparttable}
\usepackage{threeparttablex}
\usepackage[normalem]{ulem}
\usepackage{makecell}
\usepackage{xcolor}
\usepackage{tikz} % required for image opacity change
\usepackage[absolute,overlay]{textpos} % for text formatting
\usepackage{chemfig}

\floatplacement{figure}{H}

% Added by CII
\usepackage[format=hang,labelfont=bf,margin=0.5cm,justification=centering]{caption}
\captionsetup{font=small,width=0.9\linewidth,labelfont=small,textfont={small,bf}}
% End of CII addition

\usepackage{subcaption}
\newcommand{\subfloat}[2][need a sub-caption]{\subcaptionbox{#1}{#2}}

% \captionsetup[sub]{font=footnotesize}
\captionsetup[subfigure]{font=small,labelfont=small,textfont=small}

% \newcommand*{\AlignChar}[1]{\makebox[1ex][c]{\ensuremath{\scriptstyle#1}}}%

% this font option is amenable for beamer, although these are global settings
\setbeamerfont{caption}{size=\tiny}
\setbeamerfont{subcaption}{size=\tiny} % this is doubted if has any effects

\singlespacing
\definecolor{lightgrayd}{gray}{0.95}
\definecolor{skyblued}{rgb}{0.65, 0.6, 0.94}
\definecolor{oranged}{RGB}{245, 145, 200}

% define new column types (say, L, C, and R) that take their width as argument and do the following
% - Issue \raggedright, \centering, or \raggedleft to achieve the desired horizontal alignment,
% - Declare \let\newline\\ to allow to use \newline for manual line breaks within a cell (note that \centering & friends change the meaning of \\ -- this is the problem with Jake's solution),
% - Issue \arraybackslash (i.e., \let\\\tabularnewline) to allow (again) to use \\ for ending rows,
% - Issue \hspace{0pt} to allow the first word in a cell to be hyphenated.
\newcolumntype{R}[1]{>{\raggedleft\let\newline\\\arraybackslash\hspace{0pt}}p{#1}}
\newcolumntype{L}[1]{>{\raggedright\let\newline\\\arraybackslash\hspace{0pt}}m{#1}}
\newcolumntype{C}[1]{>{\centering\let\newline\\\arraybackslash\hspace{0pt}}m{#1}}

% Added by CII
%%% Copied from knitr
%% maxwidth is the original width if it's less than linewidth
%% otherwise use linewidth (to make sure the graphics do not exceed the margin)
\makeatletter
\def\maxwidth{ %
  \ifdim\Gin@nat@width>\linewidth
    \linewidth
  \else
    \Gin@nat@width
  \fi
}
\makeatother

\usepackage{background}